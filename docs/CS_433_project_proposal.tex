\documentclass{article} % EXPAND THE PREAMBLE TO SEE ALL OF THE CODE
\usepackage[utf8]{inputenc}
\usepackage{amsmath, amsfonts} 
\allowdisplaybreaks
\usepackage{amssymb}
\usepackage{esint} % for more integral symbols
\usepackage[margin=1.75in]{geometry} % [margin=1.0in]
\usepackage{graphicx} % more arguments for the \includegraphics command
\usepackage{wrapfig} 
\usepackage{textcomp} % more text symbols
\usepackage{mwe} % used for idk?
\usepackage{gensymb} % for symbols (e.g \degree and \ohm)
\providecommand{\e}[1]{\ensuremath{\times 10^{#1}}}
\usepackage{multicol} % for columns
\usepackage{float} % to better format tables
\restylefloat{table} % to better format tables
\usepackage{tabularx} % to better format tables
\usepackage{titlesec} 
\usepackage{enumitem} % allows better formatting for list environments
\usepackage{tikz} % graphs and drawings
\newcommand{\comment}[1]{} % for multiline comments
\usepackage[stretch=10]{microtype} % best package ever
\usepackage{booktabs} % for all your fancy table needs
\usepackage{environ}
\usepackage{siunitx}
\usepackage{mathrsfs}
\usepackage{cancel}
\usepackage{multirow} % allows multirow cells in tables
\usepackage{xcolor} % fancier color options
\usepackage{tikz-qtree} % trees
\usepackage{forest} % more trees (better imo)
\usepackage{mathtools}
\usepackage{hyperref}

\usepackage{contour}
\usepackage{ulem}
\renewcommand{\ULdepth}{1.8pt}
\contourlength{0.8pt}

\newcommand{\myuline}[1]{
  \uline{\phantom{#1}}%
  \llap{\contour{white}{#1}}%
}



\renewcommand{\arraystretch}{1} 

\title{CS 433 Project Proposal}
\author{Zane Globus-O'Harra, Doug Ure}
\date{\textit{14 April 2023}}

\begin{document}
\maketitle

Two-factor authentication (2FA) is the contemporary approach to
authorizing a user. The first authentication factor is the user's
password, with the second factor being an additional piece of
information that only the user could know, often a PIN or a push
notification sent to the user over a secure channel to a trusted device. However, both
PIN-2FA and push-based 2FA have some issues, which are addressed in the
paper ``2D-2FA: A New Dimension In Two-Factor Authentication'' by
Maliheh Shirvanian and Shashank
Agrawal.\footnote{\href{https://arxiv.org/abs/2110.15872}{\texttt{https://arxiv.org/abs/2110.15872}}}

Some attacks to these two common 2FA methods include shoulder surfing,
short PINs, and neglectful user approvals. The authors present a new
approach to 2FA, which they have coined ``2D-2FA.'' In this new
approach, when a user logs in with their username and password, a
unique identifier is displayed to them. The user then inputs
this same identifier on their device. A one-time PIN is generated on the
device, and transferred automatically to the server, along with the
identifier. The identifier is used in the PIN's computation, so that the
PIN is bound to a specific session. 

The user's device and the server agree on a secret key during a one-time
registration process, which is also used in the PIN computation. Once
the PIN is transferred to the server, the server authenticates the
session associated with the identifier by verifying the PIN, thereby
taking two dimensions into account (the PIN and the identifier).

For our project, we hope to replicate the results of this paper using
software. We will create programs for the server and device, make it so
that they can talk to each other, and implement the 2D-2FA requirements.
If we have time, we will also attempt to implement ``Typing-Proof,'' an
algorithm presented by Dr. Li et al. in his
paper.\footnote{Typing-Proof: Usable, Secure and Low-Cost Two-Factor
Authentication Based on Keystroke Timings, by Ximing Liu, Yingjiu Li,
Robert H. Deng} However, we think that this additional implementation
will be a bit more challenging for us, as it involves cross-referencing
audio intensity spikes with keystroke timings, and would involve
more work of figuring out how to filter audio and detect keystrokes.


\end{document}
